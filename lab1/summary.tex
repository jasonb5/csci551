\documentclass[table]{report}

\usepackage{pgfplots}
\usepackage{xcolor}
\usepackage{float}

\begin{document}

\title{Parallel Numerical Integration with Open MPI\\
Lab \#1}
\date{\today}
\author{Jason Boutte}
\maketitle

\section{Finding n}

I found n by letting x be a number divisible by 2, e.g. 4096. The approximate
integral is calculated by adding x to a variable and using it as the number
of trapezoids. The result is used to calculate the absolute relative true error,
and compared to $5.0^{-13}$. If the absolute relative true error is not less than
or equal to the value then x is added to the variable and the process is repeated.
If the error is less than or equal to the value then x is subtracted from the 
variable and x is then divided by 2. At this point we've found a minimum and the 
whole process is repeated until x is equal 1, in which case we've found a minimum 
number of trapezoids to result in an approximate integral value with 14 correct 
significant digits. The value I used for n was 2387968.

\subsection{Absolute relative error}

I used absolute relative true error. My n value of 2,387,968 resulted in an
absolute relative true error of $1.1358118166260693^{-14}$.

\section{Timings}

\begin{table}[h!]
\centering
	\rowcolors{2}{yellow}{yellow}
	\begin{tabular}{|l|l|l}
	\hline
	Run & Time (Seconds) \\ \hline
	1 & $4.494059^{-01}$ \\ \hline
	\hiderowcolors 2 & $4.506230^{-01}$ \\ \hline
	3 & $4.497890^{-01}$ \\ \hline
	4 & $4.497440^{-01}$ \\ \hline
	5 & $4.500141^{-01}$ \\ \hline
	\end{tabular}
	\caption{1 Process}
	\label{table:1}
\end{table}

\begin{table}[h!]
\centering
	\rowcolors{5}{yellow}{yellow}
	\begin{tabular}{|l|l|l}
	\hline
	Run & Time (Seconds) \\ \hline
	1 & $2.253621^{-01}$ \\ \hline
	2 & $2.248940^{-01}$ \\ \hline
	3 & $2.248309^{-01}$ \\ \hline
	4 & $2.248290^{-01}$ \\ \hline
	\hiderowcolors 5 & $2.251148^{-01}$ \\ \hline
	\end{tabular}
	\caption{2 Processes}
	\label{table:2}
\end{table}

\begin{table}[h!]
\centering
	\rowcolors{6}{yellow}{yellow}
	\begin{tabular}{|l|l|l}
	\hline
	Run & Time (Seconds) \\ \hline
	1 & $5.777979^{-02}$ \\ \hline
	2 & $5.934191^{-02}$ \\ \hline
	3 & $5.668807^{-02}$ \\ \hline
	4 & $5.980706^{-02}$ \\ \hline
	5 & $5.667305^{-02}$ \\ \hline	
	\end{tabular}
	\caption{8 Processes}
	\label{table:3}
\end{table}

\begin{table}[h!]
\centering
	\rowcolors{4}{yellow}{yellow}
	\begin{tabular}{|l|l|l}
	\hline
	Run & Time (Seconds) \\ \hline
	1 & $3.507686^{-02}$ \\ \hline
	2 & $3.340387^{-02}$ \\ \hline
	3 & $3.262115^{-02}$ \\ \hline
	\hiderowcolors 4 & $3.407717^{-02}$ \\ \hline
	5 & $3.264308^{-02}$ \\ \hline	
	\end{tabular}
	\caption{14 Processes}
	\label{table:4}
\end{table}

\begin{table}[h!]
\centering
	\rowcolors{6}{yellow}{yellow}
	\begin{tabular}{|l|l|l}
	\hline
	Run & Time (Seconds) \\ \hline
	1 & $2.458596^{-02}$ \\ \hline	
	2 & $2.370000^{-02}$ \\ \hline	
	3 & $2.407598^{-02}$ \\ \hline	
	4 & $2.467895^{-02}$ \\ \hline	
	5 & $2.308488^{-02}$ \\ \hline	
	\end{tabular}
	\caption{20 Processes}
	\label{table:5}
\end{table}

\section{Speedup and Efficiency}

\begin{figure}[H]
\centering
	\begin{tikzpicture}
		\begin{axis}[
			title={Speedup},
			axis lines=left,
			xlabel=Processes,
			ylabel=Speedup,
			xmin=0, xmax=24,
			ymin=0, ymax=24,
			xtick={0,4,8,12,16,20,24},
			ytick={0,4,8,12,16,20,24},
		]
		\addplot[
			color=blue,
			mark=square,
		]
		coordinates{(2,2.20)(8,8.72)(14,15.15)(20,21.40)};
		\end{axis}
	\end{tikzpicture}
	\label{fig:speedup}
\end{figure}

\begin{figure}[H]
\centering
	\begin{tikzpicture}
		\begin{axis}[
			title={Efficiency},
			axis lines=left,
			xlabel=Processes,
			ylabel=Efficiency,
			xmin=0, xmax=24,
			ymin=1.06, ymax=1.11,
			xtick={0,4,8,12,16,20,24},
		]
		\addplot[
			color=blue,
			mark=square,
		]
		coordinates{(2,1.098743934)(8,1.089713276)(14,1.081811296)(20,1.070092199)};
		\end{axis}
	\end{tikzpicture}
	\label{fig:efficiency}
\end{figure}

\section{Conclusion}

The speed at which you can approximate an integral using the Trapezoidal
Method greatly benefits from parallelization. By simply doubling the 
processes performing the integration the speed was doubled and by utilizing
20 processes the speed was approximately 20 times faster than a single
process. But all this speed comes at a cost, a loss of accuracy. The 
accuracy loss at 2 and 8 processes wasn't very great, we lost 1 significant
digit of the 14 we were looking at. But once the processes reached 14 and
20 the loss was far greater, losing almost 9 significant digits of the 14.
This loss of accuracy is most likely caused by the software rounding the
floating point values during the reduce call. Though our hardware is capable
of calculating floating point at high precision, the software communicates
these values at a lower precision. This explanation maps well to our 
observation. If we lose 2 significant values each time we add a process 
we can see how a few processes wouldn't have great losses but 20 processes 
each losing 2 values would add up quickly. From this lab I've seen the power
of parallelizing large problems but also witnessed the drawbacks. When 
solving problems using these tools we must keep in mind that loss in 
precision may occur. We must account for it in the design and decide
an acceptable amount of loss and weigh the benefits of speed versus accuracy.

\section{Discussion}

Utilizing 20 processes saw a huge loss in precision due to the software
rounding floating point numbers. I decided the best way to counter this
would be to have each process calculate its value at a higher precision
to negate the loss in communication. After multiple attempts I found
that around 4,400,000 trapezoids could bring my absolute relative true
error back down to an acceptable value. Using 20 processes I was now
only losing 1 significant value of the 14. While this worked for 20
processes it can be seen that increasing the number of trapezoids and
processes would still be bottlenecked by rounding.

\end{document}
